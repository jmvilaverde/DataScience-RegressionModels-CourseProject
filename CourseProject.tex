\documentclass[11pt,]{article}
\usepackage{lmodern}
\usepackage{amssymb,amsmath}
\usepackage{ifxetex,ifluatex}
\usepackage{fixltx2e} % provides \textsubscript
\ifnum 0\ifxetex 1\fi\ifluatex 1\fi=0 % if pdftex
  \usepackage[T1]{fontenc}
  \usepackage[utf8]{inputenc}
\else % if luatex or xelatex
  \ifxetex
    \usepackage{mathspec}
    \usepackage{xltxtra,xunicode}
  \else
    \usepackage{fontspec}
  \fi
  \defaultfontfeatures{Mapping=tex-text,Scale=MatchLowercase}
  \newcommand{\euro}{€}
\fi
% use upquote if available, for straight quotes in verbatim environments
\IfFileExists{upquote.sty}{\usepackage{upquote}}{}
% use microtype if available
\IfFileExists{microtype.sty}{%
\usepackage{microtype}
\UseMicrotypeSet[protrusion]{basicmath} % disable protrusion for tt fonts
}{}
\usepackage[margin=.5in]{geometry}
\usepackage{longtable,booktabs}
\usepackage{graphicx}
\makeatletter
\def\maxwidth{\ifdim\Gin@nat@width>\linewidth\linewidth\else\Gin@nat@width\fi}
\def\maxheight{\ifdim\Gin@nat@height>\textheight\textheight\else\Gin@nat@height\fi}
\makeatother
% Scale images if necessary, so that they will not overflow the page
% margins by default, and it is still possible to overwrite the defaults
% using explicit options in \includegraphics[width, height, ...]{}
\setkeys{Gin}{width=\maxwidth,height=\maxheight,keepaspectratio}
\ifxetex
  \usepackage[setpagesize=false, % page size defined by xetex
              unicode=false, % unicode breaks when used with xetex
              xetex]{hyperref}
\else
  \usepackage[unicode=true]{hyperref}
\fi
\hypersetup{breaklinks=true,
            bookmarks=true,
            pdfauthor={by jmvilaverde},
            pdftitle={Motor Trend magazine - Data analysis of influence on MPG for Automatic vs.~Manual Transmission.},
            colorlinks=true,
            citecolor=blue,
            urlcolor=blue,
            linkcolor=magenta,
            pdfborder={0 0 0}}
\urlstyle{same}  % don't use monospace font for urls
\setlength{\parindent}{0pt}
\setlength{\parskip}{6pt plus 2pt minus 1pt}
\setlength{\emergencystretch}{3em}  % prevent overfull lines
\setcounter{secnumdepth}{0}

%%% Use protect on footnotes to avoid problems with footnotes in titles
\let\rmarkdownfootnote\footnote%
\def\footnote{\protect\rmarkdownfootnote}

%%% Change title format to be more compact
\usepackage{titling}

% Create subtitle command for use in maketitle
\newcommand{\subtitle}[1]{
  \posttitle{
    \begin{center}\large#1\end{center}
    }
}

\setlength{\droptitle}{-2em}
  \title{Motor Trend magazine - Data analysis of influence on MPG for Automatic
vs.~Manual Transmission.}
  \pretitle{\vspace{\droptitle}\centering\huge}
  \posttitle{\par}
  \author{by jmvilaverde}
  \preauthor{\centering\large\emph}
  \postauthor{\par}
  \predate{\centering\large\emph}
  \postdate{\par}
  \date{Thursday, June 18, 2015}



\begin{document}

\maketitle


\subsection{Executive summary}\label{executive-summary}

With the linear regression model evaluated have concluded that, with
95\% confidence, is estimated that a manual transmission results in a
0.0457303 to 5.8259441 increase in MPG comparing to use an automatic
transmission.

\begin{center}\rule{0.5\linewidth}{\linethickness}\end{center}

\subsection{1.Initial Exploratory Data
Analysis}\label{initial-exploratory-data-analysis}

\subsubsection{\texorpdfstring{\textbf{Structure from ?mtcars and values
for factors.} \emph{(View Figure 1.Mtcars pairs
comparation.)}}{Structure from ?mtcars and values for factors. (View Figure 1.Mtcars pairs comparation.)}}\label{structure-from-mtcars-and-values-for-factors.-view-figure-1.mtcars-pairs-comparation.}

\emph{Format:} \emph{A data frame with 32 observations on 11 variables.}

\begin{longtable}[c]{@{}lll@{}}
\toprule
Variables & Units & Values\tabularnewline
\midrule
\endhead
\textbf{mpg} & \textbf{Miles/(US) gallon}\tabularnewline
cyl & Number of cylinders (4,6,8) & 4, 6, 8\tabularnewline
disp & Displacement (cu.in.)\tabularnewline
hp & Gross horsepower\tabularnewline
drat & Rear axle ratio\tabularnewline
wt & Weight (lb/1000)\tabularnewline
qsec & 1/4 mile time\tabularnewline
vs & V/S -\textgreater{} V motor or straight motor & 0, 1\tabularnewline
\textbf{am} & \textbf{Transmission (0 = automatic, 1 = manual)} & 0,
1\tabularnewline
gear & Number of forward gears & 3, 4, 5\tabularnewline
carb & Number of carburetors & 1, 2, 3, 4, 6, 8\tabularnewline
\bottomrule
\end{longtable}

Correlation between mpg and am is 0.5998324. (Closer to -1 or 1 is
stronger relationship, when is 0 implies no linear relationship).

\subsection{2.Model proposal and
analysis}\label{model-proposal-and-analysis}

\subsubsection{\texorpdfstring{2.1.Analysis of Model Initial.
\emph{(View Figure C1.Figure Summary Model
Initial.)}}{2.1.Analysis of Model Initial. (View Figure C1.Figure Summary Model Initial.)}}\label{analysis-of-model-initial.-view-figure-c1.figure-summary-model-initial.}

\begin{quote}
\(mpg_i={\beta}_0+\beta_{am}am_i\)
\end{quote}

\begin{itemize}
\itemsep1pt\parskip0pt\parsep0pt
\item
  P-value intercept and coefficients: 1.133983e-15, 2.850207e-04
  \textless{} 0.05 are good p-values.
\item
  P-value Model: 2.850207e-04. \textless{} 0.05 is a good p-value for
  the model.
\item
  \(R^2\) value: 0.3597989. \textbf{This low value indicates that model
  Initial only fits 36\% of the data.}
\end{itemize}

\paragraph{\texorpdfstring{2.2.Analysis of model Complete. \emph{(View
Figure C2.Figure Summary Model
Complete.)}}{2.2.Analysis of model Complete. (View Figure C2.Figure Summary Model Complete.)}}\label{analysis-of-model-complete.-view-figure-c2.figure-summary-model-complete.}

\begin{quote}
\(mpg =\beta_{cyl}cyl+\beta_{dips}disp+\beta_{hp}hp+\beta_{drat}drat+\beta_{wt}wt+\beta_{qsec}qsec+\beta_{vs}vs+\beta_{am}am+\beta_{gear}gear+\beta_{carb}carb\)
\end{quote}

\begin{itemize}
\itemsep1pt\parskip0pt\parsep0pt
\item
  P-value intercept and coefficients: 5.181244e-01, 9.160874e-01,
  4.634887e-01, 3.349553e-01, 6.352779e-01, 6.325215e-02, 2.739413e-01,
  8.814235e-01, 2.339897e-01, 6.652064e-01, 8.121787e-01 \textgreater{}
  0.05 are bad p-values.
\item
  P-value Model: 5.03445e-10. \textless{} 0.05 is a good p-value for the
  model.
\item
  \(R^2\) value: 0.8690158. This high value indicates that the model
  Complete is a good fit to the data.
\end{itemize}

\textbf{The p-value for intercept and coefficients indicates that model
Complete is not a good model for fit the data.}

\paragraph{\texorpdfstring{2.3.Propose and analysis of alternative model
using step R function. \emph{(View Figure C3.Figure Summary Model
Step.)}}{2.3.Propose and analysis of alternative model using step R function. (View Figure C3.Figure Summary Model Step.)}}\label{propose-and-analysis-of-alternative-model-using-step-r-function.-view-figure-c3.figure-summary-model-step.}

\begin{quote}
step(lm(mpg \textasciitilde{} ., data=mtcars)): lm, mpg
\textasciitilde{} wt + qsec + am, mtcars
\end{quote}

\begin{quote}
\(mpg =\beta_{wt}wt+\beta_{qsec}qsec+\beta_{am}am\)
\end{quote}

\begin{itemize}
\itemsep1pt\parskip0pt\parsep0pt
\item
  P-value intercept and coefficients: 1.779152e-01, 6.952711e-06,
  2.161737e-04, 4.671551e-02 \textless{} 0.05 for coefficients are good
  p-value for the model.
\item
  p-value Model: 2.038468e-12. \textless{} 0.05 is a good p-value for
  the model.
\item
  R\^{}2: 0.8496636. This high value indicates that the model Step is a
  good fit to the data.(1 means 100\% fit).
\end{itemize}

\emph{This model have a p-value \textless{} 0.05 for model and
coefficients, and a \(R^2\) value near 85\%. This model is accepted.}

\emph{(View Figure 2. Model Step plot.)}

\emph{(View Figure C4. Anova comparation between the models.)}

\begin{center}\rule{0.5\linewidth}{\linethickness}\end{center}

\subsection{Final Analysis based on information from Model
Step.}\label{final-analysis-based-on-information-from-model-step.}

\subsubsection{Is an automatic or manual transmission better for
MPG?}\label{is-an-automatic-or-manual-transmission-better-for-mpg}

\begin{quote}
The manual transmission mean is higher than automatic transmission mean
for Model Step, due to this analysis is determined that manual
transmission is better for MPG.
\end{quote}

\subsubsection{Quantify the MPG difference between automatic and manual
transmissions.}\label{quantify-the-mpg-difference-between-automatic-and-manual-transmissions.}

\begin{quote}
With 95\% confidence, we estimate that a manual transmission results in
a 0.0457303 to 5.8259441 increase in MPG comparing to use of automatic
transmission for a car with the same weight (wt) and 1/4 mile time
(qsec).
\end{quote}

\begin{center}\rule{0.5\linewidth}{\linethickness}\end{center}

\subsection{Apendix. (Note: Summary and Anova are included as
Figures).}\label{apendix.-note-summary-and-anova-are-included-as-figures.}

\emph{Figure C1.Figure Summary Model Initial.}

\begin{verbatim}
## 
## Call:
## lm(formula = mpg ~ am, data = mtcars)
## 
## Residuals:
##     Min      1Q  Median      3Q     Max 
## -9.3923 -3.0923 -0.2974  3.2439  9.5077 
## 
## Coefficients:
##             Estimate Std. Error t value Pr(>|t|)    
## (Intercept)   17.147      1.125  15.247 1.13e-15 ***
## am             7.245      1.764   4.106 0.000285 ***
## ---
## Signif. codes:  0 '***' 0.001 '**' 0.01 '*' 0.05 '.' 0.1 ' ' 1
## 
## Residual standard error: 4.902 on 30 degrees of freedom
## Multiple R-squared:  0.3598, Adjusted R-squared:  0.3385 
## F-statistic: 16.86 on 1 and 30 DF,  p-value: 0.000285
\end{verbatim}

\emph{Figure C2.Figure Summary Model Complete.}

\begin{verbatim}
## 
## Call:
## lm(formula = mpg ~ ., data = mtcars)
## 
## Residuals:
##     Min      1Q  Median      3Q     Max 
## -3.4506 -1.6044 -0.1196  1.2193  4.6271 
## 
## Coefficients:
##             Estimate Std. Error t value Pr(>|t|)  
## (Intercept) 12.30337   18.71788   0.657   0.5181  
## cyl         -0.11144    1.04502  -0.107   0.9161  
## disp         0.01334    0.01786   0.747   0.4635  
## hp          -0.02148    0.02177  -0.987   0.3350  
## drat         0.78711    1.63537   0.481   0.6353  
## wt          -3.71530    1.89441  -1.961   0.0633 .
## qsec         0.82104    0.73084   1.123   0.2739  
## vs           0.31776    2.10451   0.151   0.8814  
## am           2.52023    2.05665   1.225   0.2340  
## gear         0.65541    1.49326   0.439   0.6652  
## carb        -0.19942    0.82875  -0.241   0.8122  
## ---
## Signif. codes:  0 '***' 0.001 '**' 0.01 '*' 0.05 '.' 0.1 ' ' 1
## 
## Residual standard error: 2.65 on 21 degrees of freedom
## Multiple R-squared:  0.869,  Adjusted R-squared:  0.8066 
## F-statistic: 13.93 on 10 and 21 DF,  p-value: 3.793e-07
\end{verbatim}

\emph{Figure C3.Figure Summary Model Step.}

\begin{verbatim}
## 
## Call:
## lm(formula = mpg ~ wt + qsec + am, data = mtcars)
## 
## Residuals:
##     Min      1Q  Median      3Q     Max 
## -3.4811 -1.5555 -0.7257  1.4110  4.6610 
## 
## Coefficients:
##             Estimate Std. Error t value Pr(>|t|)    
## (Intercept)   9.6178     6.9596   1.382 0.177915    
## wt           -3.9165     0.7112  -5.507 6.95e-06 ***
## qsec          1.2259     0.2887   4.247 0.000216 ***
## am            2.9358     1.4109   2.081 0.046716 *  
## ---
## Signif. codes:  0 '***' 0.001 '**' 0.01 '*' 0.05 '.' 0.1 ' ' 1
## 
## Residual standard error: 2.459 on 28 degrees of freedom
## Multiple R-squared:  0.8497, Adjusted R-squared:  0.8336 
## F-statistic: 52.75 on 3 and 28 DF,  p-value: 1.21e-11
\end{verbatim}

\emph{Figure C4. Anova comparation between the models.}

\begin{verbatim}
## Analysis of Variance Table
## 
## Model 1: mpg ~ am
## Model 2: mpg ~ wt + qsec + am
## Model 3: mpg ~ cyl + disp + hp + drat + wt + qsec + vs + am + gear + carb
##   Res.Df    RSS Df Sum of Sq       F    Pr(>F)    
## 1     30 720.90                                   
## 2     28 169.29  2    551.61 39.2687 8.025e-08 ***
## 3     21 147.49  7     21.79  0.4432    0.8636    
## ---
## Signif. codes:  0 '***' 0.001 '**' 0.01 '*' 0.05 '.' 0.1 ' ' 1
\end{verbatim}

\includegraphics{CourseProject_files/figure-latex/pairsMtcars-1.pdf}

\begin{center}\includegraphics{CourseProject_files/figure-latex/plotModel-1} \end{center}

\end{document}
